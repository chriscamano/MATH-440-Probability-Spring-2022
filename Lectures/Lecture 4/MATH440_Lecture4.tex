\documentclass[12pt,a4paper]{article}
\usepackage[utf8]{inputenc}
\usepackage{amsmath}

\usepackage{amsfonts}
\usepackage{amssymb}
\usepackage{tikz}
\usepackage{amsmath}
\usepackage{amssymb}
\usepackage{pgfplots}
\usepackage{nccmath}
\usepackage{mathtools}
\usepackage{pgfplots}
\usepackage{mathtools,amssymb}
\usepackage{tikz}
\usepackage{xcolor}
\pgfplotsset{compat = newest}
\author{Chris Camano: ccamano@sfsu.edu}
\title{MATH 440 Lecture 4 }
\date{2/17/2022}
% Margins
\topmargin=-0.45in
\evensidemargin=0in
\oddsidemargin=0in
\textwidth=6.5in
\textheight=9.0in
\headsep=0.25in
\newcommand{\q}{\quad}
\renewcommand{\labelenumi}{\alph{enumi})}
\newcommand{\R}{\mathbb{R}}
\newcommand{\rtwo}{$\mathbb{R}^2$}
\newcommand{\C}{$\mathbb{C}$}

\begin{document}
\maketitle
\section*{Combinatorics}

\textbf{Multiplication rule and summation rule}\\
\textbf{multiplication rule}
Assume you have some operation or some object that can be chosen in n ways:

\[
  A_1,A_2,...,A_k
\]
Each with
\[
  n_1,n_2,...,n_k
\]
different ways of being selected. .

\textbf{Rule of Sum}
When you have different events A and B with n ways for A and m ways for B you can find the combinations by summing the different ways.

\[
  \textbf{or }\rightarrow \textbf{rule of sum}
\]
\[
  \textbf{and }\rightarrow \textbf{Multiplication rule}
\]
\section*{permutations}
Given an ordered arrangement of some objects of length k which are selected from a finite collection of objects (size n) is called a permutation:
\[
  nP_k=\frac{n!}{(n-k)!}
\]
\\
\section*{counting permutations with repetitions}\\
The number of ways to arrange n objects such that $n_1$ being of one kind $n_2$ being of second time.... such that $n_1+...+n_r=1$ is
\[
  N=\frac{n!}{\prod_{i=1}^r n_i!}
\]
\section*{Combinations}
\[
  nC_r=n \choose k=\frac{n!}{(n-k)!k!}
\]\\
\section*{Binomial Coefficients}
\textbf{Property of Pascal's triangle}
\[
  {n \choose k} = {n \choose {k-1}} + {n \choose {k+1}}

  \]
  \textbf{Multinomial Coefficients}
  \\
  Given
  \[
    (x_1+x_2+..+x_r)^n=\sum_{n_1}+\sum_{n_2}+...+\sum_{n_r}\frac{n!}{\prod_{i=1}^rn_i!}\prod_{j=1}^i x_j^{n_j}
  \]
  \end{document}
