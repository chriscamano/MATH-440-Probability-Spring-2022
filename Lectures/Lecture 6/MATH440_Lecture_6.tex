\documentclass[12pt]{article}
\usepackage[pdftex]{graphicx}
\usepackage{amsmath}
\usepackage{amssymb}
\pagestyle{empty}
\author{Chris Camano: ccamano@sfsu.edu}
\title{MATH 425  Lecture 6 }
\date{3/8/2022}

\topmargin -0.6in
\headsep 0.40in
\oddsidemargin 0.0in
\textheight 9.0in
\textwidth 6.5in

\newcommand{\econst}{\mathrm{e}}
\newcommand{\diff}{\mathrm{d}}
\newcommand{\dwrt}[1]{\frac{\diff}{\diff #1}}
%%%%%%Macros for 425%%%%%%%%%%%%%%%%%%%
\newcommand{\q}{\quad}
\newcommand{\tab}{\\\\}
\renewcommand{\labelenumi}{\alph{enumi})}
\newcommand{\sect}[1]{\section*{#1}}

%%%%%%Vector Spaces%%%%%%%%%%%%%%%%%%%
\newcommand{\R}{\mathbb{R}}
\newcommand{\C}{\mathbb{C}}
\newcommand{\F}{\mathbb{F}}
\newcommand{\rtwo}{\mathbb{R}^2}
\newcommand{\mxn}{{mxn}}

%%%%%%Sets and common phrases%%%%%%%%%
\newcommand{\Axb}{\textbf{Ax=b} }
\newcommand{\Axz}{\textbf{Ax=0} }
\newcommand{\dim}{\text{dim}}
\newcommand{\lc}{linear combination }
\newcommand{\let}{\text{Let }}
\newcommand{\tf}{\therefore}
%%%%%%%%%%%%%%%%%%%%%%%%%%%%%%%%%%%%%%%
\everymath={\displaystyle}

\begin{document}
\maketitle

\sect{continuous random variables}
A discrete random variable is a variable of the form:
\newcommand{\rv}{random variable }
\[
  X:\mathcal{S}\mapsto \R
\]
Where $\mathcal{S}$ is a discrete sample space
\[
  P(x_i)=P(X=x_i)=P(\{s:X(s)=x_i\})
\]
For continuous \rv this means that
\[
  \mathcal{S}=\R \text{ or } \mathcal{S}\subset \R
\]
\textbf{Example}\\
Let $y_1,y_2,..,y_n$ be a set of measurements . Let n =40. Assigning arbitrary data for these points we can construct a histogram. This is used to discretize data and visualize frequency.
\\
\textbf{frequency} The number of times an event occurs. \\
\textbf{Relative frequency (w)}
\[
  \frac{f}{\mathcal{S}}
\]
where f is frequency\\
\textbf{Density}
\[
  \frac{w}{l}
\]
where l is the length of each bin on ths histogram.\\
When you decrease the size of the bins in a progressively shrinking density function the behavior of the histogram can be described by a function;;
\\
When this is accomplished we can describe a function f(x) such that :
\[
  \int_{-\infty}^\infty f(x)dx=1
\]
\[
  f(x)\geq 0
\]
\[
  \int_a^b f(x)dx=P(a\leq y\leq b)
\]
\newcommand{\cont}{continuous }
\sect{Continuous \rv}
Let Y be a function from a sample space S to $\R$. The function Y is called a continuous random variable if there exists a function f(y) such that for all real numbers a and b such that $a<b$
\[
    P(a\leq y\leq b)=\int_a^b f(y)dy
\]
In this event f(y) is called a probability density function (pdf) for a \cont \rv\\
The function f(y) is a pdf for some \cont \rv y $\iff$ the following conditions are met.
\begin{align*}
  &f(y) \geq 0\\
  & \int_{-\infty}^\infty f(x)dx=1
\end{align*}

\sect{Common distributions}
\textbf{Uniform distribution}\\
\[
  f(y)=c\quad \forall y : a \leq y \leq b \\ \quad \quad
  f(y)=0 \text{ otherwise}
\]
\\\\
\textbf{Exponential distribution}
\[
  f(y)=\lambda e^{-\lambda y} \text{ if }y \geq 0
\]
and zero otherwise. $\lambda >0$\\
\textbf{Normal distribution}\\
\[
  f(y)=\frac{1}{\sqrt{2\pi}\sigma}e^{-\frac{(y-\mu)^2}{2\sigma^2}}
\]

\sect{cumulative distribution function}
\[
  F_y(t)=P(y \leq t)=\int_{-\infty}^tf(y) dy
\]
\[
  \frac{d}{dx}F_y(t)=f(t)
\]
\end{document}
