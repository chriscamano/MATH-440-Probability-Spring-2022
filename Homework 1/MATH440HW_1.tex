\documentclass[12pt,a4paper]{article}
\usepackage[utf8]{inputenc}
\usepackage{amsmath}

\usepackage{amsfonts}
\usepackage{amssymb}
\usepackage{tikz}
\usepackage{amsmath}
\usepackage{amssymb}
\usepackage{pgfplots}
\usepackage{nccmath}
\usepackage{mathtools}
\usepackage{pgfplots}
\usepackage{mathtools,amssymb}
\usepackage{tikz}
\usepackage{xcolor}
\pgfplotsset{compat = newest}
\author{Chris Camano: ccamano@sfsu.edu}
\title{MATH440 Homework Set 1 }
\date{1/30/2022}
% Margins
\topmargin=-0.45in
\evensidemargin=0in
\oddsidemargin=0in
\textwidth=6.5in
\textheight=9.0in
\headsep=0.25in
\newcommand{\q}{\quadd}
\renewcommand{\labelenumi}{\alph{enumi})}
\begin{document}
\maketitle
\textbf{Homework questions: 2.2.12, 2.2.16, 2.2.22,2.2.38 2.3.10, 2.3.12, 2.3.14}
\section{Chapter 2.2 Sample Spaces and the Algebra of Sets}\\
\textbf{Problem 2.2.12}
Consider the experiment of choosing coefficients for the quadratic equation $ax^2+bx+c=0$. Characterize the values of a,b,and c associated with the event A: Equation has complex roots.
\begin{align*}
  A=\{(a,b,c)|b^2-4ac<0,a,b,c \in \mathbb{R}\}
\end{align*}
\\\\
\textbf{Problem 2.2.16}
Sketch the regions in the xy plane coresponding to $A\cup B $ and $A\cap B$ if:
\begin{align*}
  A=\{(x,y)|0 <x<3,0<y<3\}\\
  B=\{(x,y)|2 <x<4,2<y<4\}
\end{align*}
\textbf{Please see attached illustration for image of region(Still working on Tikz skills)}.
\\\\
\textbf{Problem 2.2.22}
Suppose that each of the twelve letters in the word T E S S E L L A T I O N is written on a chip. Define the following events F,R and C as follows:
\begin{center}
  \text{F: Letters in the first half of the alphabet}\\
  \text{R: Letters that are repeated}\\
  \text{V: Letters that are vowels}\\
  F=\{A,E,E,I,L,L\}\\
  R=\{T,T,L,L,E,E,S,S\}\\
  V=\{E,A,I,O,E\}
\end{center}
\begin{enumerate}
  \item {\text{What is} F\cap R\cap V\\
  \q F \cap R =\{E,E\}\\
  \q (F \cap R)\cap V=\{E,E\}\\}

  \item {\text{What is} F^C\cap R\cap V^C\\
  \q F^C=\{N,O,S,S,T,T\}\\
  \q V^C=\{S,S,T,T,N,L,L\}\\
  \q R^C= \{I,O,N,A\}\\\\
  \q F^C\cap R=\{S,S,T,T\}\\
  \q (F^C \cap R)\cap V^C=\{S,S,T,T\}\\}

  \item { \text{What is} F \cap R^C \cap V\\
  \q F \cap R^C =\{I,A\}\\
  \q (F \cap R^C) \cap V =\{I,A\}}
\end{enumerate}
\\\\
\textbf{Problem 2.2.38}\\
\textbf{Please see attached illustration figure 2}.
\\\\
\textbf{Problem 2.3.10}\\
An urn contains twenty-four chips, numbered 1 through 24. one is drawn at random. Let A be the event that the number is divisible by 2 and B be the event that the number is divisible by 3. Find $P(A \cup B)$
\begin{align*}
A=\{2,4,6,8,10,12,14,16,18,20,22,24\}\\
B=\{3,6,9,12,15,18,21,24\}\\
A \cup B=\{2,3,4,6,8,9,12,14,15,16,18,20,21,22,24\}\\
P(A \cup B)= \frac{|A\cup B|}{|S|}=\frac{16}{24}=\frac{2}{3}
\end{align*}
\\\\
\textbf{Problem 2.3.12}\\
Events $A_1$ and $A_2$ are such that $A_1\cup A_2=S$ and $A_1\cap A_2=\emptyset$ find $p_2$ if $P(A_1)=p_1, P(A_2)=p_2, \text{ and } 3p_1-p_2=\frac{1}{2}$
\begin{align*}
  3p_1-p_2=\frac{1}{2}\\
  p_1+p_2=1 \text{ as }A_1 \cup A_2=S\\
  \end{align*}

\end{align*}
\[
\begin{bmatrix}
3 & -1 & \frac{1}{2} \\
1 & 1 & 1
\end{bmatrix}
\sim
\begin{bmatrix}
1 & 0 & \frac{3}{8} \\
0 & 1 & \frac{5}{8}
\end{bmatrix}
\therefore p_1=\frac{3}{8},p_2=\frac{5}{8}
\]
\textbf{Problem 2.3.14}\\
Three events A B and C are defined on a sample space S given that:
\[
  P(A)=.2 \q P(B)=.1 \q P(C)=.3
\]
What is the smallest possible value for:
\[
  P[(A \cup B \cup C}^C)]
\]
\[
  P[(A \cup B \cup C)^C)]=P(A^c \cap B^c \cap C^c)
\]
by Demorgan's Rules. Strategy: find the union of all three sets then subtract that total from the sample space to represent the complement:
\[
  P(A \cup B \cup C)= P(A)+P(B)+P(C) - P(A \cap B)- P( A \cap C)-P( B \cap C) - P(A \cap B \cap C)
\]
\begin{align*}
  P(A \cup B)= P(A)+P(B)-P(A\cap B) \therefore\\
  P(A \cup B \cup C)= \textbf{$P(A \cup B)$} +P(C)- P( A \cap C)-P( B \cap C) - P(A \cap B \cap C)
\end{align*}
The union of A and B are largest if they are disjoint therefore:
\[
  P(A \cup B ) \leq P(A)+P(B)
\]
\[
  P(A \cup B) \leq .2 +.1
\]
\[
  P(A \cup B ) \leq .3
\]
Extending the same logic if $P(A \cup B)$ is disjoint from C in all ways then the union will be at its largest.
\begin{align*}
  \textbf{Let: } P( A \cap C)=P( B \cap C) = P(A \cap B \cap C)=0
\end{align*}
therefore:
\[
  P(A \cup B \cup C) \leq P(A \cup B)+ P(C)=.3+.3
\]
\[
    P(A \cup B \cup C) \leq .6
\]
\[
P[(A \cup B \cup C)^C]=1-.6=.4
\]
Therefore the smallest $P[(A \cup B \cup C)^C]$ could possibly be is .4  as this only occurs when the union of the three events is at its largest which occurs when the three events are disjoint.
\end{document}
