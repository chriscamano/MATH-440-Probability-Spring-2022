\documentclass[12pt,a4paper]{article}
\usepackage[utf8]{inputenc}
\usepackage{amsmath}

\usepackage{amsfonts}
\usepackage{amssymb}
\usepackage{tikz}
\usepackage{amsmath}
\usepackage{amssymb}
\usepackage{pgfplots}
\usepackage{nccmath}
\usepackage{mathtools}
\usepackage{pgfplots}
\usepackage{mathtools,amssymb}
\usepackage{tikz}
\usepackage{xcolor}
\pgfplotsset{compat = newest}
\author{Chris Camano: ccamano@sfsu.edu}
\title{MATH440 Lecture 3 }
\date{2/10/2022}
% Margins
\topmargin=-0.45in
\evensidemargin=0in
\oddsidemargin=0in
\textwidth=6.5in
\textheight=9.0in
\headsep=0.25in
\newcommand{\q}{\quadd}
\renewcommand{\labelenumi}{\alph{enumi})}
\newcommand{\rtwo}{$\mathbb{R}^2$}
\newcommand{\C}{$\mathbb{C}$}
\newcommand{\tf}{\therefore}
\begin{document}
\maketitle
\section{2.5 Independence of events}
review: conditional probability.
\begin{align*}
  \text{Let s be a sample space with events }A, B \subset S\\
  P(B)>0\\
  P(A|B)=\frac{P \cap B}{P(B)} \therefore\\
  P(A \cap B)=P(A|B)P(B)
\end{align*}
\textbf{Independence}\\
Two events A and B are indepepdent if $$P(A|B)=P(A)$$ and dependent otherwise.\\\\
This is to say that the probability of A does not depend on whether or not event B occurs.\\
Two events A and B are also indepepdent if \[
  P(A \cap B)= P(A)P(B)
\]
This is because we know the following logic:\\
\begin{align*}
  P(A|B)=P(A)\\
  P(A \cap B)= P(A|B)P(B)=P(A)P(B)
\end{align*}

\textbf{Theorem}\\
Let A and B be two indepepdent events. Then $A^C, B^C$ are also indepepdent. \\
$$P(A \cap B)=P(A)P(B)$$
We have to then show the following to proof the indepepdence of the complements:\\
$$P(A^C \cap B^C)=P(A^C)P(B^C)$$
\begin{align*}
  P(A^C \cap B^C)=P((A \cup B)^C)\\
  P(A^C \cap B^C)=1-P(A \cup B)\\
  P(A^C \cap B^C)=1-[P(A)+P(B)-P(A \cap B)]\\
  P(A^C \cap B^C)=1-[P(A)+P(B)-P(A)P(B)]\\
  P(A^C \cap B^C)=1-P(A)-P(B)+P(A)P(B)]\\
  P(A^C \cap B^C)=[1-P(A)]-P(B)+P(A)P(B)]\\
  P(A^C \cap B^C)=P(A^C)-[P(B)(1-P(A)]]\\
  P(A^C \cap B^C)=P(A^C)-[P(B)(P(A^C)]]\\
  P(A^C \cap B^C)=P(A^C)[1-P(B)]\\
  P(A^C \cap B^C)=P(A^C)P(B^C)\\
\end{align*}
Indepdent Events:
\[
  P(A \cap B)=P(A)+P(B)
\]
Mutually exclusive events:
\[
  P(A \cap B)= \emptyset
\]
The major take away is that when two events are independent the intersection is the product of the two events
\textbf{Independence of more than two events}\\
Given $A_1,...,A_n$ independent events are said to be independent if for every set of indicies $i_1,...,i_k$
\[
P(A_{i_1} \cap,...,\cap A_{i_k})=P(A_{i_1})P(A_{i_2})...P(A_{i_k})
\]

\end{document}
