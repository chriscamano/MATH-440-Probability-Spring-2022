\documentclass[12pt]{article}
\usepackage[pdftex]{graphicx}
\usepackage{amsmath}
\usepackage{amssymb}
\pagestyle{empty}
\author{Chris Camano: ccamano@sfsu.edu}
\title{MATH 440 Homework 6 }
\date{3/10/2022}

\topmargin -0.6in
\headsep 0.40in
\oddsidemargin 0.0in
\textheight 9.0in
\textwidth 6.5in

\newcommand{\econst}{\mathrm{e}}
\newcommand{\diff}{\mathrm{d}}
\newcommand{\dwrt}[1]{\frac{\diff}{\diff #1}}
%%%%%%Macros for 425%%%%%%%%%
\newcommand{\q}{\quad}
\newcommand{\tab}{\\\\}
\renewcommand{\labelenumi}{\alph{enumi})}
\newcommand{\sect}[1]{\section*{#1}}

\newcommand{\R}{\mathbb{R}}
\newcommand{\C}{\mathbb{C}}
\newcommand{\F}{\mathbb{F}}
\newcommand{\Z}{\mathbb{Z}}
\newcommand{\rtwo}{\mathbb{R}^2}
\newcommand{\mxn}{{mxn}}

\newcommand{\Axb}{\textbf{Ax=b} }
\newcommand{\Axz}{\textbf{Ax=0} }
\newcommand{\dim}{\text{dim}}
\newcommand{\lc}{linear combination }
%%%%%%%%%%%%%%%%%%%%%%%%%%%%%
\everymath={\displaystyle}


\begin{document}
\maketitle

\sect{3.4.6}
\begin{proof}
  Let n be  postive integer, show that $f(y)=(n+2)(n+1)y^n(1-y), 0\leq y \leq 1$ is a pdf\\
  \begin{align*}
    &\int_0^1(n+2)(n+1)y^n(1-y)dy=1\\
    &(n+2)(n+1)\int_0^1y^n(1-y)dy=1\\
    &(n+2)(n+1)\int_0^1 y^n-y^{n+1}dy=1\\
    &(n+2)(n+1)\Big[\int_0^1 y^ndy-\int_0^1y^{n+1}dy\Big]=1\\
    &(n+2)(n+1)\Big[ \frac{y^{n+1}}{n+1}|_0^1-\frac{y^{n+2}}{n+2}|_0^1\Big]=1\\
    &(n+2)(n+1)\Big[ \frac{1}{n+1}-\frac{1}{n+2}\Big]=1\\
    &(n+2)(n+1)\Big[ \frac{n+2-n+1}{(n+1)(n+2)}\Big]=1\\
    &n+2-n+1=1\\
    &1=1
  \end{align*}
\end{proof}\\
\sect{3.4.9}
\begin{proof}
  \[
    F_Y(y)=\int_{-\infty}^yf_Y(t))dy\\
  \]
  \begin{align*}
    &\forall y>0\\
    &F_Y(y)=\int_{-1}^y1-|t|dt\\
    &\int_{-1}^01+tdt+\int_{0}^y1-tdt\\
    &F_Y(y)=(t+\frac{t^2}{2}\Big|_{-1}^0)+(t-\frac{t^2}{2}\Big|_{0}^y)\\
    &F_Y(y)=1-\frac{1}{2}+y-\frac{y^2}{2}\\
  \end{align*}
  \begin{align*}
    &\forall y\leq 0\\
    &F_Y(y)=\int_{-1}^y1-|t|dt\\
    &\int_{-1}^y1+tdt\\
    &F_Y(y)=t+\frac{t^2}{2}\Big|_{-1}^y\\
    &F_Y(y)=\frac{1}{2}+y+\frac{y^2}{2}\\
  \end{align*}
  Thus, $F_Y(y)$=0 $\quad\quad\forall y<-1$\\ $F_Y(y)=\frac{1}{2}+y+\frac{y^2}{2} \quad\quad\forall y, -1\leq y \leq 0$ \\$F_Y(y)=1-\frac{1}{2}+y-\frac{y^2}{2}\quad\quad \forall y, 0<y\leq 1$ \\and 1 for all y greater than one
\end{proof}\\
\sect{3.4.14}
\begin{proof}
  \begin{align*}
    &F_Y(y)=\int_0^yte^{-t}dt\\
    &let u=t,du =1, dv=e^{-t}, v=-e^{-t}\\
    &F_Y(y)=\int_0^yte^{-t}dt=-te^{-t}+\int e^{-t}dt\\
    &F_Y(y)=\int_0^yte^{-t}dt=-te^{-t}-e^{-t}\\
    &F_Y(y)=-te^{-t}-e^{-t}\Big|_0^y\\
    &F_Y(y)=-ye^{-y}-e^{-y}+1\\
  \end{align*}
\end{proof}\\
\end{document}
