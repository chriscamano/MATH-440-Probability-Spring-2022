\documentclass[12pt]{article}
\usepackage[pdftex]{graphicx}
\usepackage{amsmath}
\usepackage{amssymb}
\pagestyle{empty}
\author{Chris Camano: ccamano@sfsu.edu}
\title{MATH 440  Homework 9 }
\date{4/14/2022}

\topmargin -0.6in
\headsep 0.40in
\oddsidemargin 0.0in
\textheight 9.0in
\textwidth 6.5in

\newcommand{\econst}{\mathrm{e}}
\newcommand{\diff}{\mathrm{d}}
\newcommand{\dwrt}[1]{\frac{\diff}{\diff #1}}
%%%%%%Macros for 425%%%%%%%%%%%%%%%%%%%
\newcommand{\q}{\quad}
\newcommand{\tab}{\\\\}
\renewcommand{\labelenumi}{\alph{enumi})}
\newcommand{\sect}[1]{\section*{#1}}

%%%%%%Vector Spaces%%%%%%%%%%%%%%%%%%%
\newcommand{\R}{\mathbb{R}}
\newcommand{\C}{\mathbb{C}}
\newcommand{\F}{\mathbb{F}}
\newcommand{\rtwo}{\mathbb{R}^2}
\newcommand{\mxn}{{mxn}}

%%%%%%Sets and common phrases%%%%%%%%%
\newcommand{\Axb}{\textbf{Ax=b} }
\newcommand{\Axz}{\textbf{Ax=0} }
\newcommand{\dim}{\text{dim}}
\newcommand{\lc}{linear combination }
\newcommand{\let}{\text{Let }}
\newcommand{\tf}{\therefore}
%%%%%%%%%%%%%%%%%%%%%%%%%%%%%%%%%%%%%%%
\everymath={\displaystyle}


\begin{document}
\maketitle
\begin{center}
  \textbf{3.8:2,6,7}\\\\
  \textbf{3.9:4,10,14,20}
\end{center}

\sect{Problem 3.8.2}
\begin{proof}
\begin{align*}
  &f_Y(y)=\frac{1}{2}(1+y),\quad -1\leq y \leq y\\
  &W=-4Y+7\\
\end{align*}

\[
  f_W(w)=\frac{1}{|a|}f_Y(\frac{w-b}{a})
\]
\begin{align*}
  &f_W(w)=\frac{1}{|a|}f_Y(\frac{w-b}{a})=\frac{1}{8}(1+\frac{w-7}{-4})=\\
  &f_W(w)=\frac{w-11}{-32}\quad 11\leq w \leq 3
\end{align*}
\end{proof}
\sect{Problem 3.8.6}
\begin{proof}
One interesting way of proving this that I found when researching ways to show independence of random variables is through the examination of the covariance. If two independent random variables are independent then their covariance should be zero since there is no relationship between them (page 188). This is to say that:
\[
  cov(W,(X+Y))=0
\]
I am not able to complete this proof at this time and have worked with another class mate to produce the following:
\begin{align*}
  &f_{WV}=P(v\leq w , X+Y \leq W)\\
  &\int _{-\infty}^v\int_{-\infty}^w\int_{-\infty}^{w-x}f_V(v)f_X(x)f_Y(y)dvdxdy\\
  &\int _{-\infty}^vf_V(v)f_{XY}(x,y)dv=F_V(v)F_{XY}(w)
\end{align*}
\end{proof}
\sect{Problem 3.8.7}
\begin{proof}
\[
  W=Y^2
\]
Show that W has pdf:
\[
  f_W(w)=\frac{1}{2\sqrt{w}}f_Y(\sqrt{w})
\]
\begin{align*}
  &F_W(w)=P(W\leq w)\\
  &W=Y^2 \therfore\\
  &F_W(w)=P(Y\leq \sqrt{w})\quad \text{as $Y^2\leq w$ implies $Y\leq \sqrt{w}$ }\\
\end{align*}
Note that we are now describing the CDF of Y instead of W after the substitution (very cool)\\
\begin{align*}
  &P(Y\leq \sqrt{w})=F_Y(\sqrt{w})\\
  &\frac{d}{dw}F_Y=f_Y \therefore\\
  &\frac{d}{dw}F_Y(\sqrt{w})=f_Y(\sqrt{w})\frac{d}{dw}\sqrt{w} \quad \text{by chain rule}\\
  &=\frac{1}{2\sqrt{w}}f_Y(\sqrt{w})
\end{align*}
Please let me know if I missed something with this proof it felt slightly wrong to make the assumption that the deriviative of the CDF with respect to w gives us what we want here.
\end{proof}
\sect{Problem 3.9.4}
\begin{proof}
\[
  E(X+Y)=4X+6Y
\]
\begin{align*}
  &E(4X+6Y)=4E(X)+6E(Y)\\
  &=4(np_x)+6(np_y)\\
  &=4(10(.3))+6(10(.4))\\
  &=4(3)+6(4)=36
\end{align*}
The expected score is 36.
\end{proof}
\sect{Problem 3.9.10}
\begin{proof}
\[
  E(X^2+Y^2)
\]
Since we are uniformly dist. over[0,1] then the marginals are:

\[
  f_X(x)=1 \land f_Y(y)=1
\]
\[
    f_X(x)f_Y(y)=f_{XY}(xy)=1
\]
so we know the variables are independent thus:
\begin{align*}
  & E(X^2+Y^2)=E(X^2)+E(Y^2)\\
  &\int_0^1x^2dx+\int_0^1y^2dy\\
  &\frac{1}{3}+\frac{1}{3}=\frac{2}{3}
\end{align*}
\end{proof}
\sect{Problem 3.9.14}
\begin{proof}
Show that $Cov(aX+b,cY+d)=acCov(X,Y)$ $\forall a,b,c,d$\\
\begin{align*}
    &Cov(aX+b,cY+d)=E((aX+b)(cY+d))-E(aX+b)E(cY+d)\\
    &Cov(aX+b,cY+d)=E(acXY+daX+bcY+bd)-\Big([aE(x)+b][cE(y)+d]\Big)\\
    &Cov(aX+b,cY+d)=E(acXY+daX+bcY+bd)-\Big(aE(x)cE(y)+adE(x)+bcE(y)+bd\Big)\\
    &Cov(aX+b,cY+d)=acE(XY)+daE(X)+bcE(Y)+bd-acE(X)E(Y)-adE(X)+bcE(y)-bd\\
    &Cov(aX+b,cY+d)=acE(XY)-acE(X)E(Y)\\
    &Cov(aX+b,cY+d)=ac(E(XY)-E(X)E(Y)\\
    &Cov(aX+b,cY+d)=acCov(X,Y)
\end{align*}

\end{proof}
\sect{Problem 3.9.20}
\begin{proof}
\[
  X\sim binom(p_x,n)
\]
\[
  X\sim binom(p_y,m)
\]
\begin{align*}
  &E(4X+6Y)=4E(X)+6E(Y)\\
  &E(4X+6Y)=4np_x+6mp_y
\end{align*}
\begin{align*}
  &Var(4X+6Y)=16Var(X)+36Var(Y)\\
  &Var(4X+6Y)=16np_xp_x^c+36mp_yp_y^c
\end{align*}
\end{proof}

\end{document}
