\documentclass[12pt,a4paper]{article}
\usepackage[utf8]{inputenc}
\usepackage{amsmath}

\usepackage{amsfonts}
\usepackage{amssymb}
\usepackage{tikz}
\usepackage{amsmath}
\usepackage{amssymb}
\usepackage{pgfplots}
\usepackage{nccmath}
\usepackage{mathtools}
\usepackage{pgfplots}
\usepackage{mathtools,amssymb}
\usepackage{tikz}
\usepackage{xcolor}
\pgfplotsset{compat = newest}
\author{Chris Camano: ccamano@sfsu.edu}
\title{MATH440 Lecture 2 }
\date{2/3/2022}
% Margins
\topmargin=-0.45in
\evensidemargin=0in
\oddsidemargin=0in
\textwidth=6.5in
\textheight=9.0in
\headsep=0.25in
\newcommand{\q}{\quadd}
\renewcommand{\labelenumi}{\alph{enumi})}
\begin{document}
\maketitle

\section{2.4 Conditional Probability }\\
\textbf{Example 1}: Assume you roll a dice once:
\[
  S=\{1,2,3,4,5,6\}
\]
Let event A be the number 6 occurs thus:
\[
  P(A)=\frac{1}{6}
\]
Let event B be the event even  number occurs.
\[
P(B)=\frac{1}{2}
\]
\[
  P(A|B)=\frac{1}{3}
\]
Using conditional probability one constructs a subset of the sample space. The condition probability notation is read probability of A given B occurs.
\begin{align*}
  \text{Let A and B be any two events defined on a sample space such that }P(B) > 0\\
\end{align*}
\[
    P(A|B)=\frac{P(A \cap B)}{P(B)}
\]
\subsection{Multiplication rule}
As an extension of this property we define the multiplication rule as:

\[
  P(A\cap B)=P(B|A)P(A)
\]
Or
\[
  P(A\cap B)=P(A|B)P(B)
\]
Due to the commutatitivy of sets.\\
Additionally:
\[
  P(A\cap B \cap C)= P(A)P(B|A)P(C|B \cap A)
\]
For a collection of three sets.
\\
Given \[
  A_1,A_2,...,A_n
\]
\[
  P\Big(\bigcap_{i=1}^nA_n\Big)=P(A_1)P(A_2|A_1)P(A_3|A_1\cap A_2)...P\Big(A_n|\bigcap_{i=1}^{n-1}A_i\Big)
\]
\subsection{Baye's Theorem }
\end{document}
